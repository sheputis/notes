\documentclass[11pt,a4paper]{article}

% Modern font handling (LuaLaTeX)
\usepackage{fontspec}
\setmainfont{Latin Modern Roman}

\newfontfamily\EmojiFont{Noto Emoji} % monochrome (most reliable in PDFs)
\newcommand{\emoji}[1]{{\EmojiFont #1}}

% Typography
\usepackage{microtype}

% Math
\usepackage{amsmath, amssymb, amsthm}
\usepackage{mathtools}

% Physics helpers
\usepackage{physics}

% Links
\usepackage{hyperref}

\usepackage{graphicx}

\newcommand{\note}[1]{%
  \par\medskip
  \noindent\textbf{Note:} #1
  \par\medskip
}

\setlength{\parindent}{0pt}

\title{Can we send information faster than light?}
\date{\today}

\begin{document}

\maketitle

\note{
Everything you are about to see is just pure linear algebra. The notation might seem
unfamiliar, but its nothing fancy. It is all just matrices and vectors.
}

\section{Motivation}

\begin{figure}[h]
  \centering
  \includegraphics[width=0.6\textwidth]{figures/electron-positron.png}
  \caption{An entangled electron-positron pair gets created and they are being sent in
           opposite directions to planet A and B respectively}
\end{figure}

Look I added an emoji \emoji{😄}. Anyways, let's say we have an entangled pair of particles,
for example an electron and a positron whose spins are entangled. Alice receives the electron
on planet A and Bob receives the positron on planet B. If either 'observe' the state of the spin
of their particle, the other particle will immediately also collapse into some state as prescribed
by the entangled quantum state. For example, if the overall state was

\[
\ket{\psi} = a\ket{01} + b\ket{10}
\]

(Let $0$ represent the spin down state and $1$ the up state)

we see that there are only two possible states:
\begin{itemize}
  \item Either A is in spin down and B is in spin up with probability $\abs{a}^2$
  \item Or A is in spin up and B is in spin down with probability $\abs{b}^2$
\end{itemize}

All well and good so far, but here is a catch. Bob does not know it, but Alice has a crush
on him. Alice is finally ready to invite Bob on a date. There is only one problem, apart
from the fact that they live on different planets: Clara also likes Bob, and she is currently
in the same city as Bob. She is also about to drive to Bob to proclaim her admiration for his
quantum knowledge. Alice knows this for some reason and is devastated. Planet A is 10 lightyears
away. If she will send a Whatsapp message to Bob now, by the time Bob reads that Alice thinks
his quantum logic is hot, it will be 10 years later and he will already have 5 kids with Clara.

\begin{figure}[h]
  \centering
  \includegraphics[width=0.6\textwidth]{figures/clara_admires_bob.png}
  \caption{Clara admires Bob for his quantum knowldege and Bob likes quantum knowldege}
\end{figure}

Its late at night and Matteo has made sure the fire is heating the cabin really good. He is expecting
Alice to come through the door any minute now. He had finally managed to ask her out. And he is
not wasting any time: if he did not take this seriously he would take her to the movies or some simple
thing like that. But with Alice he wants to hit the home run right away. And a cabin trip alla Matteo
is a nuclear weapon in the romantic world. An unrestistable smell of spagetthi carbonara has taken over
the cabin. Matteo' special secret is to not use the timer for the spagetthi. He has to feeeel the spagetthi
and only strike it out of the water when he feels deep in his soul that the time is high, just like he
will do with Alice later that night. While rehearsing his romantic lines in front of an old mirror, he
could not tell if the twinkles in his eyes actually were a reflection of the hundred candles he lit up
everywhere or if he is just that excited.

\begin{figure}[h]
  \centering
  \includegraphics[width=0.6\textwidth]{figures/matteo_anticipating.png}
  \caption{Golly gee can't wait!}
\end{figure}

Matteo was just about to finish pointing his index fingers towards the mirror anyways as the door
opened with a sudden crackle followed by a gust of cold winter air finding its way into the room.
Alice has finally arrived, but to Matteo's horror, she was crying! The carbonara was already long
cold and the last more stubborn candles were holding on for dear life in an attempt to prolong the night.
So was Matteo's mood as he was listening to Alice going on and on about Bob and sobbing. Only if she
could somehow send Bob a message before Clara gets to him and charms him.

\begin{figure}[h]
  \centering
  \includegraphics[width=0.6\textwidth]{figures/matteo_disappointed.png}
  \caption{Oh Bob, my admiration for you is faster than light}
\end{figure}

She is frantically telling Matteo that she knows quite a bit of quantum physics because her full time
job is appearing in quantum thought experiments. She tells him she has recently received an electron from
a planet far far away (actually, a whole bunch of them). All she needs to do is
to observe its spin state and bang: the positron instantly assumes the opposite state. That is instant!
Surely she can somehow make use of this fact to send a message to Bob before its too late! That is not, however,
what Matteo wanted to hear tonight, the situation seems hopeless... the night is ruined. Or is it? Matteo
has one last trick up his sleeve. As a professor in Category Theory, he is going to use his mathematical
powers to prove to Alice that even though the collapse on the quantum state is instant, she will not be
able to send information to Bob faster than the speed of light, and that she is better off forgetting Bob
and enjoing the carbonara with him tonight.

\section{The Math}

Before we proceed with the arguments and the proof, let's define the basic mathematical objects that model
the quantum systems that we are interested in.

\subsection{How do we describe a quantum state?}

The electron-positron system lives in a composite Hilbert space

\[
\mathcal{H}_{AB} = \mathcal{H}_A \otimes \mathcal{H}_B
\]

Its density matrix is:

\[
 \rho_{AB} = \sum_{i, j, k, l} \rho_{ijkl} \ket{i}_A\bra{j}_A \otimes \ket{k}_B\bra{l}_B =
 \sum_{i, j, k, l} \rho_{ijkl} \ket{i}\bra{j} \otimes \ket{k}\bra{l}
\]
where $i, j \in \mathcal{X} $ and  $k, l \in \mathcal{Y} $, where $\mathcal{X}$ and $\mathcal{Y}$
are the sets that index the states in the two Hilbert spaces. In our special case of the electron and
positron spins we have

\[
 \mathcal{X} = \{0, 1\}
\]

\[
 \mathcal{Y} = \{0, 1\}
\]
which means that the electron spin can be either in up or down state, and the same for the positron.
For the sake of familiarizing ourselves with these objects let's look at how the particular density
matrix in the case of Alice and Bob looks like, before proceeding with the proof for a more general
case. We saw earlier the vector representing the entangled quantum state looks like this

\[
\ket{\psi}_{AB} = a\ket{01} + b\ket{10}
\]

For pure states, the density matrix can be expressed as an outer product of itself

\begin{align}
\rho_{AB}
&= \ket{\psi}\bra{\psi} \\
&= (a\ket{01} + b\ket{10})
   (a^*\bra{01} + b^*\bra{10}) \\
&= |a|^2 \ket{01}\bra{01}
 + ab^* \ket{01}\bra{10}
 + a^*b \ket{10}\bra{01}
 + |b|^2 \ket{10}\bra{10}
\label{eq:rho_ab}
\end{align}

But what can we do with this global description? In practice, we are often interested only in a
subsystem, such as a single particle, rather than the full composite system. For example if Bob
is going to make a measurement on the positron, he is only interested in the statistics of what
can come out of the positron: basically just the stuff in the Hilbert space $\mathcal{H}_B$.
He does not care about the electron that is not in his posession.
To describe the subsystem alone, we introduce its reduced density matrix $\rho_{B}$. This object contains
all information needed to compute the probabilities of measurement outcomes for that subsystem.
The reduced density matrix is obtained from the global density matrix by taking the
partial trace over the degrees of freedom of all other subsystems. So Bob is interested
in the reduced density matrix of his positron's spin $\rho_{B}$

\begin{equation}
    \rho_{B} = \mathrm{Tr}_{A}(\rho_{AB}) = \sum_{i \in \mathcal{X}} \bra{i}_A\rho_{AB}\ket{i}_A
\end{equation}

We substitute the expression for $\rho_{AB}$ from ~\eqref{eq:rho_ab} and we get

\begin{equation}
    \rho_{B} = |a|^2 \ket{1}\bra{1} + |b|^2 \ket{0}\bra{0}
\label{eq:rho_b}
\end{equation}

So Bob's positron can be measured to be in spin up state with probability $|a|^2$ and down state with
probability $|b|^2$. An interesting observation to make is that this is a mixed
state. Meaning, it is mathematically equivalent to a classical statistical ensemble description: If
someone prepares a whole set of quantum systems where some number of them is just $\ket{0}$ and some
number is $\ket{1}$, and they forgot to put a label on them, so that person only knows how many of them
is in one and in the other, then if they gift me one of the systems, they can only tell me which state
that system is with a classical probability. Meaning, in that classical ensemble scenario, we know for sure
that each system is in some definite state, but there is a statistical ignorance that we can encode with a
density matrix. And that density matrix will actually be exactly the expression ~\eqref{eq:rho_b}. But
interestingly, in our case with Alice and Bob, we know for a fact that the positron spin is not in some
definite state, yet the density matrix is the same, so from the point of view of making measurements locally,
a person not knowing the story behind how this state was prepared would not be able to infer whether the
state was in some classical statistical ensemble or if it was due to an entanglement with some other spin
somewhere else.

\subsection{How do we manipulate a quantum state?}

Quantum states are of course not static in time, so what makes a quantum state change?
What is the mathematical description of state evolution?
For closed quantum systems, all physically valid time evolutions can be represented
by unitary transformations,

\begin{equation}
    \ket{\psi(t)} \mapsto U(t)\ket{\psi(t)}
\end{equation}

or, equivalently,

\begin{equation}
    \rho(t) \mapsto U(t)\rho(t) U^{\dagger}(t)
\end{equation}

where $U^{\dagger}$ denotes the adjoint (Hermitian conjugate) of $U$, and the unitary transformations
are defined as follows

\begin{equation}
    U^{\dagger}U = UU^{\dagger} = \mathbb{I}
\end{equation}

Ok that is basically it, so far so good, we have, in theory, everything we need to attack the
problem and help Matteo to save the night.

\section{The Arguments}

Now the idea is to translate the physical arguments and hence the physical problem into a mathematical
formulation. Then we can solve the math problem.

The physical hypothesis is that Alice can manipulate the electron spin on planet A and that it somehow
affects the probabilities for measurement of the positron spin states on planet B.

\note{
  If this is possible, then Alice can encode information by skewing the measurement probabilities
  in the positron. She could for example do it by sending some laser light on the electron.
}

Mathematically, Alice manipulating the electron spin state would mean a unitary transformation
on the subsystem $A$.

\begin{equation}
  U = U_A \otimes \mathbb{I}_B
\end{equation}

So the problem can be formulated as such:
\begin{itemize}
  \item
    We determine the reduced density matrix $\rho_B$ by taking the partial trace over $A$
    on the global state $\rho_{AB}$.
    \[
      \rho_B = \mathrm{Tr}_{A}(\rho_{AB})
    \]
  \item
    We apply the unitary transformation on the subsystem $A$ and get the state $\rho_{AB}'$.
    \[
      \rho_{AB}' = U\rho_{AB}U^{\dagger}
    \]
    Then we also determine the reduced density matrix $\rho_B'$.
    \[
      \rho_B' = \mathrm{Tr}_{A}(\rho_{AB}')
    \]
  \item
    If $\rho_{B} = \rho_{B}'$, then the unitary transformation has no effect on the reduced
    density matrix $\rho_{B}$, meaning that Alice can do nothing to affect the measurement
    probabilities for Bob, which is a sad day for Bob, but what a great night for Matteo.
\end{itemize}

\section{The Proof}

It is worth stressing what this proof does and does not say. Formally, it is a statement about operators
on tensor-product Hilbert spaces: a unitary acting on subsystem $A$ leaves the reduced state of subsystem
$B$ unchanged. To turn this into a physical statement, one uses the standard quantum-mechanical postulate
that the reduced density matrix contains all information about measurement outcomes accessible to an observer
on $B$. Under this interpretation—which is strongly supported by experiment—the result means that local
operations on $A$ cannot influence measurements performed on $B$.
Also, the proof is valid for any arbitrary number of states, meaning $\mathcal{X}$ and $\mathcal{Y}$
are free to be any set or subset of natural numbers

\subsection{Reduced Density Matrix before transformation}

\begin{align}
  \rho_{B} &= \mathrm{Tr}_{A}(\rho_{AB}) = \sum_{m \in \mathcal{X}} \bra{m}_A\rho_{AB}\ket{m}_A \\
           &= \sum_{m} \bra{m}_A(\sum_{i, j, k, l} \rho_{ijkl} \ket{i}\bra{j} \otimes \ket{k}\bra{l}) \ket{m}_A \\
           &= \sum_{m, i, j, k, l} \rho_{ijkl} \bra{m}\ket{i}\bra{j}\ket{m} \ket{k}\bra{l} \\
\end{align}

Since $\ket{i}$ are orthonormal basis vectors, the Kronecker delta function follows

\begin{equation}
  \bra{i}\ket{j} = \delta_{ij}
\end{equation}

Named after Kronecker, mostly known for bullying Cantor, the founder of set theory and the guy that
discovered multiple types of infinities, into clinical depression \emoji{💀}. Now we substitute it into
our expression.

\begin{align}
  \rho_{B} &= \sum_{m, i, j, k, l} \rho_{ijkl} \delta_{mi}\delta_{jm} \ket{k}\bra{l} \\
           &= \sum_{m, k, l} \rho_{mmkl} \ket{k}\bra{l}
\label{eq:rho_b_untransformed}
\end{align}

\subsection{Transformed Density Matrix}

Now we apply a unitary transformation on the original density matrix.

\begin{align}
  \rho_{AB}'
  &= U \rho_{AB} U^{\dagger} \\
  &= U_A \otimes \mathbb{I}_B (\sum_{i, j, k, l} \rho_{ijkl} \ket{i}\bra{j} \otimes \ket{k}\bra{l}) U_A^{\dagger} \otimes \mathbb{I}_B \\
  &= \sum_{i, j, k, l} \rho_{ijkl} U_A\ket{i}\bra{j}U_A^{\dagger} \otimes \ket{k}\bra{l}
\end{align}

\subsection{Reduced Density Matrix after transformation}

And finally we take the partial trace of $\rho_{AB}'$ over subsystem $A$ and cross our fingers for Matteo.

\begin{align}
  \rho_{B}' &= \mathrm{Tr}_{A}(\rho_{AB}') \\
            &= \sum_{m} \bra{m}\rho_{AB}'\ket{m} \\
            &= \sum_{m} \bra{m}(\sum_{i, j, k, l} \rho_{ijkl} U_A\ket{i}\bra{j}U_A^{\dagger} \otimes \ket{k}\bra{l})\ket{m} \\
            &= \sum_{m, i, j, k, l} \rho_{ijkl} \bra{m}U_A\ket{i}\bra{j}U_A^{\dagger}\ket{m} \ket{k}\bra{l}
\label{eq:rho_b_transformed}
\end{align}

Now let's stop here for a moment and before we proceed, let us find a relation for the components
of the unitary matrices from its definition $U^{\dagger}U = \mathbb{I}$. This will allow us to substitute
for it in the expression above.

A transformation $U$ in $\ket{i}$ basis can be written as

\begin{equation}
  U = \sum_{i,j} \bra{i}U\ket{j}\ket{i}\bra{j}
\end{equation}

From that follows

\begin{align}
  U^{\dagger}U &= \sum_{i,j} \bra{i}U^{\dagger}\ket{j}\ket{i}\bra{j} \sum_{k,l} \bra{k}U\ket{l}\ket{k}\bra{l} \\
               &= \sum_{i,j,k,l} \bra{i}U^{\dagger}\ket{j}\bra{k}U\ket{l}\ket{i}\bra{j}\ket{k}\bra{l} \\
               &= \sum_{i,j,k,l} \bra{i}U^{\dagger}\ket{j}\bra{k}U\ket{l} \delta_{jk}\ket{i}\bra{l} \\
               &= \sum_{i,m,l} \bra{i}U^{\dagger}\ket{m}\bra{m}U\ket{l} \ket{i}\bra{l}
\end{align}

Now this expression has to equal an identity matrix $U^{\dagger}U = \mathbb{I}$

\begin{align}
  \sum_{i,m,l} \bra{i}U^{\dagger}\ket{m}\bra{m}U\ket{l} \ket{i}\bra{l} = \delta_{il}\ket{i}\bra{l}
\end{align}

And from this follows

\begin{align}
  \sum_{m} \bra{i}U^{\dagger}\ket{m}\bra{m}U\ket{l} = \delta_{il}
\end{align}

Or, if we rename the indices to fit with our previous expressions, this is equivalent to

\begin{align}
  \sum_{m} \bra{m}U\ket{i} \bra{j}U^{\dagger}\ket{m} = \delta_{ij}
\end{align}

We now plug that back into ~\eqref{eq:rho_b_transformed} and we get

\begin{align}
  \rho_{B}' &= \sum_{m, i, j, k, l} \rho_{ijkl} \bra{m}U_A\ket{i}\bra{j}U_A^{\dagger}\ket{m} \ket{k}\bra{l} \\
            &= \sum_{i, j, k, l} \rho_{ijkl} \delta_{ij} \ket{k}\bra{l} \\
            &= \sum_{m, k, l} \rho_{mmkl} \ket{k}\bra{l}
\end{align}

And this is exactly the expression ~\eqref{eq:rho_b_untransformed}, hence

\begin{equation}
  \rho_{B}' = \rho_{B}
\end{equation}

And hence a unitary transformation on subsystem $A$ does not change the reduced density matrix of
the subsystem $B$. Hence that means that Alice has no way of sending a letter of admiration to Bob!

\begin{figure}[h]
  \centering
  \includegraphics[width=0.6\textwidth]{figures/matteo_proved.png.png}
  \caption{Irrefutable proof means the night is saved}
\end{figure}

\end{document}
