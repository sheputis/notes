\documentclass[11pt,a4paper]{article}

% Modern font handling (LuaLaTeX)
\usepackage{fontspec}
\setmainfont{Latin Modern Roman}

\newfontfamily\EmojiFont{Noto Emoji} % monochrome (most reliable in PDFs)
\newcommand{\emoji}[1]{{\EmojiFont #1}}

% Typography
\usepackage{microtype}

% Math
\usepackage{amsmath, amssymb, amsthm}
\usepackage{mathtools}

% Physics helpers
\usepackage{physics}

% Links
\usepackage{hyperref}

\usepackage{graphicx}

\newcommand{\note}[1]{%
  \par\medskip
  \noindent\textbf{Note:} #1
  \par\medskip
}

\setlength{\parindent}{0pt}

\title{Set Theory Notes}
\date{\today}

\begin{document}

\maketitle
\tableofcontents
\tableofcontents

\note{
...
}

\section{Thoughts}

\section{Definitions}

\subsection{Injective Functions}

A function $f: X \to Y$ is injective if

\[
  \forall a, b \in X: a \neq b \implies f(a) \neq f(b)
\]

or the contraposition

\[
  \forall a, b \in X: f(a) = f(b) \implies a = b
\]

\subsection{Surjective Functions}

A function $f: X\to Y$ is surjective if

\[
  \forall b \in Y,\; \exists a \in X : f(a) = b
\]

\subsection{Bijective Functions}

A function $f: X \to Y$ is bijective if it is both injective and surjective.
In this case, there exists an inverse function $f^{-1} : Y \to X$.

\subsection{Cardinality}

Cardinality answers the question of whether two sets $X$ and $Y$ are of the same size.
If there exists an injective mapping $f: X \to Y$, then

\begin{equation}
   \abs{X} \leq \abs{Y}
\end{equation}

If there exists a surjective mapping $f: X \to Y$, then

\begin{equation}
   \abs{X} \geq \abs{Y}
\end{equation}

This then implies that a bijective map $f: X \to Y$ implies

\begin{equation}
   \abs{X} = \abs{Y}
\end{equation}

\subsection{Power Set}

An important kind of set is a power set. If we have a set $X$, then its power
set $\mathcal{P}(X)$ is a set containing all subsets of $X$, including the empty set and the
entire set $X$.

\[
  \mathcal{P}(X) = \{\, Y \mid Y \subseteq X \, \}
\]

\section{Proof that some infinities are bigger than others}


\subsection{Cantor's Theorem: $\lvert X\rvert < \lvert\mathcal{P}(X)\rvert$}

Suppose towards a contradiction that there is a 'one to one' map (a bijection)

\[
  f: X \to \mathcal{P}(X)
\]

As per (insert a
reference to the proof that a one to one mapping implies that the sets are of the same size), this
would imply that the two sets are of the same size.

The idea is to show that there exists a set that breaks the initial premise that we assumed to
be true: that there exists such a bijection $f$.

Before doing so we take notice of the fact that the mapping $f$ can happen in two ways:

\begin{equation}
    x \in f(x)
    \label{eq:uninteresting_property}
\end{equation}

\begin{equation}
    x \notin f(x)
    \label{eq:interesting_property}
\end{equation}

\subsubsection{Detour To No Contradiction}

Actually the second property ~\eqref{eq:interesting_property} is the interesting one, but for the sake of exploring,
let us look at the first property ~\eqref{eq:uninteresting_property} and construct a set of all elements that satisfy that property

\[
\mathcal{B} = \{\, x \in X \mid x \in f(x) \,\}
\]

So now we want to find an $x_i$ such that $f(x_i) = \mathcal{B}$. There are two possibilities, either
$x_i \in \mathcal{B}$ or $x_i \notin \mathcal{B}$.

\begin{itemize}
  \item If $x_i \in \mathcal{B}$, then there is no contradiction because the constraint in the definition
        of $\mathcal{B}$ allows it.
  \item If $x_i \notin \mathcal{B}$, then there is also no contradiction because since $x_i$ is now not
        pointing to a set where it itself is an element, then due to the definition of $\mathcal{B}$, it
        is automatically also not part of it.
\end{itemize}


\subsubsection{The Property Leading To Contradiction}

Now back to the interesting property ~\eqref{eq:interesting_property}: let's construct a set that satisfies it.
A set of all $x$ that do not map to an element of $\mathcal{P}(X)$ that contains itself.

\[
\mathcal{D} = \{\, x \in X \mid x \notin f(x) \,\}
\]


The question is, can we find an $x_i$ such that $f(x_i) = \mathcal{D}$
There are two possibilities for it, either $x_i \in \mathcal{D}$ or $x_i \notin \mathcal{D}$.

\begin{itemize}
  \item If $x_i \in \mathcal{D}$, then, by definition, $x_i$ indexes a set that it itself is a part of.
        That is a contratiction, because $\mathcal{D}$ is defined as a set of elements that point to a
        set that does not contain itself.
  \item The other option is that $x_i \notin \mathcal{D}$. If that is the case, then by definition of
        $\mathcal{D}$, $x_i$ must be a member of $\mathcal{D}$, which is a contradiction.
\end{itemize}

\subsubsection{}

But wait, another assumption was that sets do have sizes... i need to prove that too...
Because if the above proves that they are not the same size, and we also show that X is not smaller than P(X),
then the only possibility left that P(X) is greater than X. However the main assumption here again is that the
sets do have sizes that can be compared. Finite set sizes can be compared? Ok, but can infinitely big sets be
compared?
\end{document}