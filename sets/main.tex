\documentclass[11pt,a4paper]{article}

% Modern font handling (LuaLaTeX)
\usepackage{fontspec}
\setmainfont{Latin Modern Roman}

\newfontfamily\EmojiFont{Noto Emoji} % monochrome (most reliable in PDFs)
\newcommand{\emoji}[1]{{\EmojiFont #1}}

% Typography
\usepackage{microtype}

% Math
\usepackage{amsmath, amssymb, amsthm}
\usepackage{mathtools}

% Physics helpers
\usepackage{physics}

% Links
\usepackage{hyperref}

\usepackage{graphicx}

\newcommand{\note}[1]{%
  \par\medskip
  \noindent\textbf{Note:} #1
  \par\medskip
}

\setlength{\parindent}{0pt}

\title{Set Theory Notes}
\date{\today}

\begin{document}

\maketitle
\tableofcontents
\tableofcontents

\note{
...
}

\section{Thoughts}

\section{Proof that some infinities are bigger than others}

\subsection{Power Set}

An important kind of set is a power set. If we have a set $X$, then its power
set $\mathcal{P}(X)$ is a set containing all subsets of $X$, including the empty set and the
entire set $X$.

\[
  \mathcal{P}(X) = \{\, Y \mid Y \subseteq X \, \}
\]

\subsection{Cantor's Theorem: $\lvert X\rvert < \lvert\mathcal{P}(X)\rvert$}

Suppose towards a contradiction that there is a 'one to one' map (a bijection)

\[
  f: X \to \mathcal{P}(X)
\]

As per (insert a
reference to the proof that a one to one mapping implies that the sets are of the same size), this
would imply that the two sets are of the same size.

The idea is to show that there exists a set that breaks the initial premise that we assumed to
be true: that there exists such a bijection $f$.

Before doing so we take notice of the fact that the mapping $f$ can happen in two ways:

\begin{itemize}
  \item $x \in f(x)$
  \item $x \notin f(x)$
\end{itemize}

\subsubsection{Detour To No Contradiction}

Actually the second property is the interesting one, but for the sake of exploring, let us look at the first
property and construct a set of all elements that satisfy that property

\[
\mathcal{B} = \{\, x \in X \mid x \in f(x) \,\}
\]

So now we want to find a $x_i$ such that $f(x_i) = \mathcal{B}$. There are two possibilities, either
$x_i \in \mathcal{B}$ or $x_i \notin \mathcal{B}$.

\begin{itemize}
  \item If $x_i \in \mathcal{B}$, then it fits with the definition of $\mathcal{B}$, because
        it is defined as a set with all elements that point to itself.
\end{itemize}

\[
\mathcal{D} = \{\, x \in X \mid x \notin f(x) \,\}
\]

\subsubsection{The Property Leading To Contradiction}

The question is, can we find a $x_i$ such that $f(x_i) = \mathcal{D}$
There are two possibilities for , either $\mathcal{D}$ is indexed by an $x_i$
that is inside $D$ or outside.

\begin{itemize}
  \item If $x_i \in \mathcal{D}$, then, by definition, $x_i$ indexes a set that it itself is a part of.
        That is a contratiction, because $\mathcal{D}$ is defined as a set of elements that point to A
        set that does not contain itself.
  \item The other option is that $x_i \notin \mathcal{D}$. If that is the case, then by definition of
        $\mathcal{D}$, $x_i$ must be a member of $\mathcal{D}$, which is a contradiction.
\end{itemize}

But wait, another assumption was that sets do have sizes... i need to prove that too...
Because if the above proves that they are not the same size, and we also show that X is not smaller than P(X),
then the only possibility left that P(X) is greater than X. However the main assumption here again is that the
sets do have sizes that can be compared. Finite set sizes can be compared? Ok, but can infinitely big sets be
compared?
\end{document}